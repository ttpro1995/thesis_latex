\chapter{Background}

\section{Sentiment Analysis}
In the nutshell, sentiment analysis is to determine whether the opinion about a specific product, event, organization is positive or negative.
\subsection{The need for sentiment analysis}
Every business need feedback from customer. Feedback help company know their product strength and weakness, drive business strategy. For example, fix the flaw in product, target new customer segment or halt a campaign. Traditional, company would survey their customer to collect feedback.  The feedback collected from direct survey are limited for many reason, such as low population.

With the growth of social network and e-commerce, more and more people post their opinion online via blog, social network (Facebook, Twitter), e-commerce site (Amazon, eBay). Company want to take advantage of these data. Thus, a automate system to analysis opinion are in demand.

\subsection{The challenges}
Lorem ipsum dolor sit amet, consectetur adipiscing elit, sed do eiusmod tempor incididunt ut labore et dolore magna aliqua. Ut enim ad minim veniam, quis nostrud exercitation ullamco laboris nisi ut aliquip ex ea commodo consequat. Duis aute irure dolor in reprehenderit in voluptate velit esse cillum dolore eu fugiat nulla pariatur. Excepteur sint occaecat cupidatat non proident, sunt in culpa qui officia deserunt mollit anim id est laborum.

\subsection{Different levels in sentiment analysis}
\subsubsection{Document-level}
Document-level sentiment analysis is to determine whether a document (usually a full review) about a specific entity (a product, location, service, ...) is positive or negative. For example, \cite{pang2002thumbs} perform document-level sentiment analysis on movies review data.
\subsubsection{Sentence-level}
Sentence-level sentiment analysis is to determine whether a sentence expressed positive, negative. In this thesis, we focus our study on sentence-level sentiment analysis \cite{liu2012sentiment}.
\subsubsection{Aspect-level}
Aspect-level  sentiment analysis purpose is to determine opinion, whether positive or negative, against specific aspect of entity \cite{liu2012sentiment}.

\section{Feedforward neural network}
Feedforward neural network is a network whether unit in the graph does not form a cycle \cite{Goodfelloetal2016}. 

Each feedforward neural network consist of one or multiple layer. Each layer contain one or more neurons (network nodes). Each neuron consist of a linear activation or non-linear activation function. 

\subsection{Singlelayer perceptron}
Singlelayer perceptron is a feedforward neural network only have one layer of neuron unit.  Singlelayer perceptron cannot solve some problem such as XOR.
[PICTURE GOES HERE]


\subsection{Multilayer perceptron}
Multilayer perceptron (MLP) is a feedforward neural network consist of 2 or more layer. First layer called input layer. Last layer called output layer. All layer between input and output layer are hidden layer.
[PICTURE GOES HERE]


\section{Convolution Neural Network}
Lorem ipsum dolor sit amet, consectetur adipiscing elit, sed do eiusmod tempor incididunt ut labore et dolore magna aliqua. Ut enim ad minim veniam, quis nostrud exercitation ullamco laboris nisi ut aliquip ex ea commodo consequat. Duis aute irure dolor in reprehenderit in voluptate velit esse cillum dolore eu fugiat nulla pariatur. Excepteur sint occaecat cupidatat non proident, sunt in culpa qui officia deserunt mollit anim id est laborum.
\subsection{Architect}
Lorem ipsum dolor sit amet, consectetur adipiscing elit, sed do eiusmod tempor incididunt ut labore et dolore magna aliqua. Ut enim ad minim veniam, quis nostrud exercitation ullamco laboris nisi ut aliquip ex ea commodo consequat. Duis aute irure dolor in reprehenderit in voluptate velit esse cillum dolore eu fugiat nulla pariatur. Excepteur sint occaecat cupidatat non proident, sunt in culpa qui officia deserunt mollit anim id est laborum.
\subsection{Back propagation}
Lorem ipsum dolor sit amet, consectetur adipiscing elit, sed do eiusmod tempor incididunt ut labore et dolore magna aliqua. Ut enim ad minim veniam, quis nostrud exercitation ullamco laboris nisi ut aliquip ex ea commodo consequat. Duis aute irure dolor in reprehenderit in voluptate velit esse cillum dolore eu fugiat nulla pariatur. Excepteur sint occaecat cupidatat non proident, sunt in culpa qui officia deserunt mollit anim id est laborum.
\section{Recurrent Neural Network}
Lorem ipsum dolor sit amet, consectetur adipiscing elit, sed do eiusmod tempor incididunt ut labore et dolore magna aliqua. Ut enim ad minim veniam, quis nostrud exercitation ullamco laboris nisi ut aliquip ex ea commodo consequat. Duis aute irure dolor in reprehenderit in voluptate velit esse cillum dolore eu fugiat nulla pariatur. Excepteur sint occaecat cupidatat non proident, sunt in culpa qui officia deserunt mollit anim id est laborum.
\subsection{RNN}
Lorem ipsum dolor sit amet, consectetur adipiscing elit, sed do eiusmod tempor incididunt ut labore et dolore magna aliqua. Ut enim ad minim veniam, quis nostrud exercitation ullamco laboris nisi ut aliquip ex ea commodo consequat. Duis aute irure dolor in reprehenderit in voluptate velit esse cillum dolore eu fugiat nulla pariatur. Excepteur sint occaecat cupidatat non proident, sunt in culpa qui officia deserunt mollit anim id est laborum.
\subsection{LSTM}
Lorem ipsum dolor sit amet, consectetur adipiscing elit, sed do eiusmod tempor incididunt ut labore et dolore magna aliqua. Ut enim ad minim veniam, quis nostrud exercitation ullamco laboris nisi ut aliquip ex ea commodo consequat. Duis aute irure dolor in reprehenderit in voluptate velit esse cillum dolore eu fugiat nulla pariatur. Excepteur sint occaecat cupidatat non proident, sunt in culpa qui officia deserunt mollit anim id est laborum.
\subsection{Back propagation}
Lorem ipsum dolor sit amet, consectetur adipiscing elit, sed do eiusmod tempor incididunt ut labore et dolore magna aliqua. Ut enim ad minim veniam, quis nostrud exercitation ullamco laboris nisi ut aliquip ex ea commodo consequat. Duis aute irure dolor in reprehenderit in voluptate velit esse cillum dolore eu fugiat nulla pariatur. Excepteur sint occaecat cupidatat non proident, sunt in culpa qui officia deserunt mollit anim id est laborum.
\section{Distributed Representations of Words}
\subsection{Definition}
Lorem ipsum dolor sit amet, consectetur adipiscing elit, sed do eiusmod tempor incididunt ut labore et dolore magna aliqua. Ut enim ad minim veniam, quis nostrud exercitation ullamco laboris nisi ut aliquip ex ea commodo consequat. Duis aute irure dolor in reprehenderit in voluptate velit esse cillum dolore eu fugiat nulla pariatur. Excepteur sint occaecat cupidatat non proident, sunt in culpa qui officia deserunt mollit anim id est laborum.
\subsection{work2vec}
Lorem ipsum dolor sit amet, consectetur adipiscing elit, sed do eiusmod tempor incididunt ut labore et dolore magna aliqua. Ut enim ad minim veniam, quis nostrud exercitation ullamco laboris nisi ut aliquip ex ea commodo consequat. Duis aute irure dolor in reprehenderit in voluptate velit esse cillum dolore eu fugiat nulla pariatur. Excepteur sint occaecat cupidatat non proident, sunt in culpa qui officia deserunt mollit anim id est laborum.
\subsection{Glove}
Lorem ipsum dolor sit amet, consectetur adipiscing elit, sed do eiusmod tempor incididunt ut labore et dolore magna aliqua. Ut enim ad minim veniam, quis nostrud exercitation ullamco laboris nisi ut aliquip ex ea commodo consequat. Duis aute irure dolor in reprehenderit in voluptate velit esse cillum dolore eu fugiat nulla pariatur. Excepteur sint occaecat cupidatat non proident, sunt in culpa qui officia deserunt mollit anim id est laborum.
\subsection{Different methods to produce word embedding}
Lorem ipsum dolor sit amet, consectetur adipiscing elit, sed do eiusmod tempor incididunt ut labore et dolore magna aliqua. Ut enim ad minim veniam, quis nostrud exercitation ullamco laboris nisi ut aliquip ex ea commodo consequat. Duis aute irure dolor in reprehenderit in voluptate velit esse cillum dolore eu fugiat nulla pariatur. Excepteur sint occaecat cupidatat non proident, sunt in culpa qui officia deserunt mollit anim id est laborum.
\section{Parse tree in NLP}
Parse tree is a syntactic representation of a sentence. In this thesis, we work on two type of parse tree, Constituency Parse Tree and Dependency Parse Tree.
\subsection{Constituency Parse Tree}
Constituency Parser breaks sentences into smaller phases. The root node represent a sentences. Children node represent a phase, or word constitute parent node. Each leaf node contain a word, or a punctuation. Each inner node contain a phrase. Each node are Part-of-speech tag labeled.

Constituency Parse Tree are construct based on Chomsky Phase Structure \cite{chomsky2002syntactic}. Given a list of rule, derive sentences following the rule. Constituency Parse Tree illustrate the derivation.


% \begin{equation}
% \label{eq:crule}
% \begin{aligned}
% &S \leftarrow NP + VP + .  \\
% &NP \leftarrow PRP  \\
% &NP \leftarrow DT + NN  \\
% &VP \leftarrow VBP + NP\\
% &VRP \leftarrow I \\
% &DT \leftarrow the \\
% &NN \leftarrow cat \\
% &VBP \leftarrow feed \\
% \end{aligned}
% \end{equation}
For example, we have the following rule:
\begin{enumerate}[label=(\roman*)]
	\item $S \leftarrow NP + VP + .$
	\item $NP \leftarrow PRP $
	\item $NP \leftarrow DT + NN$
	\item $VP \leftarrow VBP + NP$
	\item $VRP \leftarrow I$
	\item $DT \leftarrow the$
	\item $NN \leftarrow cat$
	\item $VBP \leftarrow feed$
\end{enumerate}
We apply rule to "I feed the cat". Result of derivation in Table \ref{ifeedmycat}  where roman on right column indicate the rule is use to derive previous statement. The derivation is illustrate in fig \ref{fig:ifeedthecatconstituency}

% \begin{equation}
% \label{eq:catparse}
% \begin{aligned}
% &S \\
% &NP + VP + .\\
% & PRP + VP + . \\
% & PRP + VBP + NP + . \\
% & PRP + VBP + DT + NN + . \\
% & I + VBP + DT + NN + . \\
% & I + feed + DT + NN + . \\
% & I + feed + my + NN + . \\
% & I + feed + my + cat + . \\
% \end{aligned}
% \end{equation}

\begin{table}[H]
	\centering
	\begin{tabular}{ll}
	S	&  \\
	NP + VP + .	& (i) \\
	PRP + VP + .	& (ii) \\
	PRP + VBP + NP + .	& (iv)  \\
	PRP + VBP + DT + NN + .	& (iii) \\
	I + VBP + DT + NN + .	&  (v) \\
	I + feed + DT + NN + .	&  (viii) \\
	I + feed + the + NN + .	&  (vi) \\
	I + feed + my + cat + .	& (vii)
	\end{tabular}
\caption{I feed my cat derivation}
\label{ifeedmycat}
\end{table}




\begin{figure}[H]
	\centering
	\includegraphics[width=0.5\linewidth]{figure/ifeedthecatconstituency}
	\caption[Constituency Parse Tree]{Constituency Parse Tree}
	\label{fig:ifeedthecatconstituency}
\end{figure}





\subsection{Dependency Parse Tree}
Dependency Parse Tree represent dependency relationship of each word in sentences. As oppose to Constituency Parse Tree, each node in Dependency Parse Tree contain one word.


\section{Recursive Neural Network}
Lorem ipsum dolor sit amet, consectetur adipiscing elit, sed do eiusmod tempor incididunt ut labore et dolore magna aliqua. Ut enim ad minim veniam, quis nostrud exercitation ullamco laboris nisi ut aliquip ex ea commodo consequat. Duis aute irure dolor in reprehenderit in voluptate velit esse cillum dolore eu fugiat nulla pariatur. Excepteur sint occaecat cupidatat non proident, sunt in culpa qui officia deserunt mollit anim id est laborum.
\subsection{RNN: Recursive Neural Network}
Lorem ipsum dolor sit amet, consectetur adipiscing elit, sed do eiusmod tempor incididunt ut labore et dolore magna aliqua. Ut enim ad minim veniam, quis nostrud exercitation ullamco laboris nisi ut aliquip ex ea commodo consequat. Duis aute irure dolor in reprehenderit in voluptate velit esse cillum dolore eu fugiat nulla pariatur. Excepteur sint occaecat cupidatat non proident, sunt in culpa qui officia deserunt mollit anim id est laborum.
\subsection{MV-RNN: Matrix-Vector RNN}
Lorem ipsum dolor sit amet, consectetur adipiscing elit, sed do eiusmod tempor incididunt ut labore et dolore magna aliqua. Ut enim ad minim veniam, quis nostrud exercitation ullamco laboris nisi ut aliquip ex ea commodo consequat. Duis aute irure dolor in reprehenderit in voluptate velit esse cillum dolore eu fugiat nulla pariatur. Excepteur sint occaecat cupidatat non proident, sunt in culpa qui officia deserunt mollit anim id est laborum.
\subsection{RNTN:Recursive Neural Tensor}
Lorem ipsum dolor sit amet, consectetur adipiscing elit, sed do eiusmod tempor incididunt ut labore et dolore magna aliqua. Ut enim ad minim veniam, quis nostrud exercitation ullamco laboris nisi ut aliquip ex ea commodo consequat. Duis aute irure dolor in reprehenderit in voluptate velit esse cillum dolore eu fugiat nulla pariatur. Excepteur sint occaecat cupidatat non proident, sunt in culpa qui officia deserunt mollit anim id est laborum. Network
\section{Programming Framework}
\subsubsection{RNTN:Recursive Neural Tensor}
Lorem ipsum dolor sit amet, consectetur adipiscing elit, sed do eiusmod tempor incididunt ut labore et dolore magna aliqua. Ut enim ad minim veniam, quis nostrud exercitation ullamco laboris nisi ut aliquip ex ea commodo consequat. Duis aute irure dolor in reprehenderit in voluptate velit esse cillum dolore eu fugiat nulla pariatur. Excepteur sint occaecat cupidatat non proident, sunt in culpa qui officia deserunt mollit anim id est laborum. Network
\subsection{Torch}
Torch \footnote{http://torch.ch/} is Lua scientific computing framework. Torch support high performing matrix calculation via multi-dimensional array call Tensor. Torch are built with C/C++, CUDA backend. Torch author choose Lua because Lua works well with C/C++ \cite{collobert2011torch7}.  Thus, Torch is high performing and support GPU. Torch have neural network package (nn) package. Computation graph must be define before forward pass.
A simple, single linear layer network can be easily defined with few line of code (see listing \ref{lst:torchlinear}).

\begin{lstlisting}[caption={Simple linear layer in Torch},label={lst:torchlinear}, language={[5.1]Lua}]
-- simple y = Ax + b linear layer
l = nn.Linear(2,3)
-- forward pass
x = torch.Tensor(2)
y = l:forward(x) -- vector dimension of 3
\end{lstlisting}

However, when a model need multiple module, such as multilayer perceptron (MLP), these module must be put into container. Figure \ref{fig:nncontainer} illustrates on function of each nn container . In order to construct two-layer perception (eq \ref{eq:mlp}), linear, tanh and softmax module must be packed into sequential module (see listing \ref{lst:torchmlp}).

\begin{equation}
\label{eq:mlp}
\begin{aligned}
&h = tanh(W_1*x + b_1) \\
&y = softmax(W_2*h + b2)
\end{aligned}
\end{equation}


\begin{lstlisting}[caption={MLP in Torch},label={lst:torchmlp}, language={[5.1]Lua}]
model = nn.Sequential()
model:add(nn.Linear(2,3))
model:add(nn.Tanh())
model:add(nn.Linear(3,5))
model:add(nn.SoftMax())
-- forward
x = torch.Tensor(2)
y = model:forward(x)
\end{lstlisting}

Torch provide nngraph package support build more complicate model. For example, define MLP in (eq \ref{eq:mlp}) use nngraph (see listing \ref{lst:torchnngraph})

\begin{lstlisting}[caption={MLP using nngraph},label={lst:torchnngraph}, language={[5.1]Lua}]
model = nn.Sequential()
model:add(nn.Linear(2,3))
model:add(nn.Tanh())
model:add(nn.Linear(3,5))
model:add(nn.SoftMax())
-- forward
x = torch.Tensor(2)
y = model:forward(x)
\end{lstlisting}

Sample code on training a model, see Appendix \ref{lst:torchtrain}

\subsection{Theano}
Theano \footnote{\url{http://deeplearning.net/software/theano/}} is a deep learning library on Python. It basic function is similar to Torch: matrix calculation, support GPU. Theano is define-and-run schema, which a computer graph must be built before it is executed.

\begin{lstlisting}[caption={Define function in Theano},label={lst:theanof}, language={python}]
x = T.dmatrix('x')
y = T.dmatrix('y')
z = x + y
f = function([x, y], z)
f([[1, 1], [2, 2]], [[3, 3], [4, 4]])
# result [[4, 4], [6, 6]]
\end{lstlisting}

Comparing to Torch7, Theano are slower on most benchmark \cite{collobert2011torch7}. Theano does not provide nice template like linear layer. Thus, model must be defined from equation. It give researcher more control over mathematics aspect but cause more trouble for beginner. A sample code for MLP \ref{lst:theanomlp}. One more problem is that the 'define-and-run' scheme does not suitable for recursive neural network due to recompile the computation graph each training sample take time.

\subsection{Pytorch}
PyTorch uses same backend as Torch. However, PyTorch specially designed for Python. Pytorch have pre-define module (Linear layer, Convolution layer) like Torch. However, Pytorch does not require to pack model into container. In Pytorch a network are defined in forward-pass thanks to Dynamic Neural Networks feature. Therefore, user can use Python control flow to define a network. For example, one can use for loop to run recurrent neural network (see listing \ref{lst:pytorchrnn}) .The features allows us to implement Recursive Neural Network for NLP, which the network change for every sample, much more easier.

\begin{lstlisting}[caption={RNN},label={lst:pytorchrnn}, language={python}]
import torch
import torch.nn as nn
rnn = nn.RNNCell(10, 20)
seq_len = 10
input_dim = 100
hidden_dim = 150
input = Variable(torch.randn(seq_len, 1, input_dim))
hx = Variable(torch.zeros(1, hidden_dim))
output = []
for i in range(6):
    hx = rnn(input[i], hx)
    output.append(hx)
\end{lstlisting}

We also implement treelstm from original Torch7 \footnote{\url{https://github.com/stanfordnlp/treelstm}} sentiment classification task in PyTorch and publish on Github \footnote{\url{https://github.com/ttpro1995/TreeLSTMSentiment}}.

We choose PyTorch because:
\begin{itemize}
	\item Dynamic Neural Networks feature works well on data sequence with different length
	\item Intuitive framework
	\item Easy to install and run on CUDA
\end{itemize}
